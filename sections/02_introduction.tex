\section{Introduction} [1000]\\
Stroke occurs in LOTS n? people worldwide and n people in England [citation]. Patients can often be left with disabilities from the stroke (include frequency and level of disability) [citation] which can also be an onwards heavy burden to the National Health Service (NHS) resource use [citation]. A stroke that is caused by a clot (ischaemic stroke) can be treated with a clot busting drug, thrombolysis, if the patient is eligible and it is detected soon enough [citation]. NICE approved use of this drug in 2000? for patients that are (list the eligibility criteria) [citation] based primarily (is this true?) on the results of a meta analysis from clinical trials performed n years ago [citation emberson]. In YEAR? NHS England set a national target for thrombolysis use of 20\% [citation]. National thrombolysis rates have yet to reach this rate, it has more-often remained (and currently is) around 11\% [citation]. A well documented side effect of thrombolysis is that it can cause a catastrophic bleed [citation], and so caution is required to give this drug to those patients that have the best likelihood of receiving benefit from it (and to avoid giving it to patients that are more likely to dis-benefit from it). Resources have been used to focus on increasing the national thrombolysis rate and understanding the barriers to do so [citation]. It has been identified that some clinicians either do not trust the meta analysis results, or (based on the polarised and uncertain outcome from thrombolysis) some clinicians will only consider giving thrombolysis to patients that match the patient cohort that were included in the clinical trials [citation].

This leads us to question: Are these higher levels of caution in selecting the patients to give thrombolysis to warranted, and do they benefit the outcome for the patients, on a population level?

Here we conduct analysis on a national stroke observational dataset to predict the effect of thrombolysis on the likelihood of a patient having a good outcome at time of discharge. We can also analyse the effect of thrombolysis on a wider cohort of patients than were included in the clinical trials (using data recorded for patients that fall outside of the clinical trial cohort, but still received thrombolysis).