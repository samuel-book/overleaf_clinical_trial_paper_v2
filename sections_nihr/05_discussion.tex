\section{Discussion}

%%%%%% INCLUDE %%%%%%

% Summarise main points

We built explainable machine learning models based on comprehensive prospective national stroke registry data capable of predicting disability outcome, at any given disability threshold, with high performance. The most influential factors that favoured being at or below any given disability threshold were lower pre-stroke disability, lower stroke severity, younger age, and use and speed of thrombolysis. The stroke team attended also had an effect on achieving any disability threshold, and that effect reduced with increasing disability at discharge. This could be due to 1) some hospitals discharging patients earlier with more disability e.g. with more community rehabilitation available, 2) effects of other hospital treatments on outcomes e.g. better/worse stroke unit care, or 3) hospitals assessing disability at discharge differently. From our model we cannot speculate further, but by including stroke team in the model we can adjust the model for these effects, allowing a clearer view of other patient features affecting outcome.

The model allows for counterfactual analysis, predicting outcomes with or without thrombolysis for any patient. In this analysis we focussed on predicting outcomes without thrombolysis, for those that actually received thrombolysis. In this way we investigate the effect of thrombolysis in those that did actually receive it. Using SHAP values, which explain the contribution of feature values to model prediction, we can also isolate the effect of thrombolysis. We found thrombolysis improved the odds of being at or below any given disability threshold, but the effect of thrombolysis was lower, and decayed to no-effect sooner, for outcomes of mRS $\leq$5 and $\leq$6. We found that the effect of thrombolysis was present for both mild-moderate and severe strokes, but with a larger effect in severe strokes. The findings from this study were remarkably similar to the meta-analysis of clinical trials \cite{emberson_effect_2014}, which extrapolate back to a 0.69 log odds (equivalent to an odds ratio of 2.0) improvement of being mRS 0-1 at stroke onset, with no effect after 378 minutes (6.3 hours). Our maximum theoretical improvement in log odds of achieving mRS 0-1 was a little higher at 0.90 (equivalent to an odds ratio of 2.5), with a decline to no-effect at 328 minutes (5.5 hours), indicating a somewhat steeper decline in effectiveness with time in our observational real-world study when compared to the original randomised data.  This is likely to be due to the licencing conditions for alteplase, in that the meta-analysis included randomised trials out to 6 hours from stroke onset \cite{emberson_effect_2014}, whereas the overwhelming majority of patients treated in the UK over the six-year study period were restricted to guideline-based eligibility within 4.5 hours \cite{intercollegiate_stroke_working_party_national_2016}.

% Compare to other work

The findings from this study were similar to the meta-analyses of clinical trials \cite{emberson_effect_2014}, which extrapolate back to a 0.69 log odds (equivalent to an odds ratio of 2.0) improvement of being mRS 0-1 at stroke onset, with no effect after 378 minutes (6.3 hours). Our maximum theoretical improvement in log odds improvement of being mRS 0-1 was a little higher, at 0.90 (equivalent to an odds ratio of 2.5), with a decline to no-effect at 328 minutes (5.5 hours).

Because of the size of the data available in the nationwide registry SSNAP, we are able to build more complex models than those derived from clinical trials, investigating stroke outcomes and benefit/disbenefit across a range of patient subgroups extending beyond the strict eligibility of randomised trials, and including comparisons of decision-making and outcomes between stroke teams.  This has, for example, allowed us to address the additional effect on outcomes from lower-thrombolysing teams adopting the less conservative practices in the use of thrombolysis found in higher thrombolysing teams. Companion papers to this paper explore these questions further \cite{pearn_are_2024, pearn_identifying_2024}. We have shown elsewhere that the majority of variation in thrombolysis rates between hospitals in the UK is accounted for by differences in clinical decision making, particularly around ‘less than ideal’ candidates for thrombolysis \cite{allen_use_2022, allen_using_2022, pearn_identifying_2024}.  The data in the current study show that in such patients there is a systematic disposition towards benefit at any level of disability outcome that supports the greater use of thrombolysis among patients not typically eligible for the original randomised trials.

% Clinical significance

The clinical significance of this work is that we have confirmed the net benefit of thrombolysis in a very large prospective national stroke registry that encompasses patients outside the strictly defined eligibility of the original randomised trials, and used sophisticated machine learning with SHAP to isolate the effect of thrombolysis from other patient characteristics that influence outcomes.  This provides significant reassurance that in real-world clinical practice, the anticipated benefits of thrombolysis are being delivered, even when treatment is deployed in a wider spectrum of patients than those represented in the randomised trials.

In summary, applying explainable machine learning to an observational data set demonstrated that the effectiveness of thrombolysis in the \textit{real world} appears to be at least as good as the clinical trials indicated. Our results should give stroke clinicians more confidence that the beneficial effect of thrombolysis is seen in real treatment populations. The size of the observational data set allows for more detailed analysis of the benefit of thrombolysis in subgroups of patients.

\subsection{Study limitations}

Though we are using a very large data set, our study necessarily has some limitations:

\begin{itemize}
    \item A group of more severely affected stroke patients who proceeded to receive thrombectomy has been excluded from the analysis, although during the six year study period endovascular intervention was increasing from a low base in the UK, and increased from 0.7\% to 2.0\% of all stroke.
    \item There is always a risk of inherent biases of an observational sample. Though our model has good accuracy, it remains possible that there are confounding patient characteristics that have an influence on the decision to treat, or the patient outcome, that are not measured in SSNAP.
    \item We have used the mRS at hospital discharge as our primary outcome, whereas the trials used the mRS at 90 days.  We know that the two are highly correlated \cite{elhabr_predicting_2021}, though individual patients may get better or worse between discharge and 90 days (possibly reflecting other aspects of care, such as community rehabilitation programmes).
\end{itemize}

\subsection{Further work}

Considering recent suggestions that thrombolysis should not be used in mild stroke \cite{braksick_thrombolysis_2024, zhang_intravenous_2024}, a more in-depth study of the observed effect of thrombolysis mild stroke, combining machine learning with causal inference methods such as propensity-score adjustment \cite{rosenbaum_central_1983} or target trial emulation \cite{hernan_target_2022}, would add to the evidence base around in this contentious group of patients.